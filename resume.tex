\documentclass{resume} 


\name
    {Зайцев Даниил Александрович}
\address
    {+7(923)4108063 | \href{mailto:zdanil34@gmail.com}{{\faEnvelope}  zdanil34@gmail.com} | Томск, Томская обл.} 
\address{
    \href{https://t.me/danzay228}{{\faPaperPlane} danzay228}
    \href{https://github.com/danzay228}{{\faGithub} danzay228}
    \href{www.danzay228.github.io}{{\faHome} danzay228.github.io} 
    \href{https://linkedin.com/in/danzay228/}{{\faLinkedin} danzay228}
}

\begin{document}

\begin{rSection}{}
{{\bf Инженер программист} с 2+ годами стажа в разработке ПО для встроенных и десктопных решений. \\ Интересна {\bf backend} или {\bf devops} разработка на полный рабочий день.}
\end{rSection}

\begin{rSection}{Образование}
    {\bfМагистр технической физики}, Томский Политехнический Университет \hfill {2016 - 2018}\\
    {\bfБакалавр технической физики}, Томский Политехнический Университет \hfill {2012 - 2016}\\
\end{rSection}

\begin{rSection}{Навыки}
\begin{tabular}{ @{} >{\bfseries}l @{\hspace{6ex}} l }
    Языки программирования  & Python, Rust, Bash, C, SQL \\
    Фреймворки              & FastApi, Django Rest Framework \\
    Базы данных             & PostgreSQL \\
    Средства разработки     & Git, Docker, VS Code, Neovim \\
    ОС                      & Linux \\ 
    Языки                   & Русский, Английский
\end{tabular} \\
\end{rSection}

\begin{rSection}{Опыт}
\textbf{Инженер-программист} \hfill Февраль 2020 - Июнь 2022\\
\href{https://tesart.ru/}{ООО НПК "ТЕСАРТ"} \\ Проектирование радаров и измерительных комплексов \hfill \textit{Томск, Томская обл.}
 \begin{itemize}
    \itemsep -3pt {} 
     \item Запуск и отладка периферийных устройств (Analog Device, Texas Instruments) с использованием различных интерфейсов UART, SPI, I2C.
     \item Разработка ПО для SoC (ARM-FPGA) на arm-linux и сервисов для управления периферийным оборудованием.
    \item Десктопные клиентские приложения для управления оборудованием.
 \end{itemize}

\end{rSection} 

\begin{rSection}{Проекты}
\vspace{-1.25em}
\item \textbf{Радар "Аргус".} {Радар с цифровой антенной решеткой.}
\item \textbf{САР Радар.} {Радар с синтезируемой апертурой.}
\item \textbf{Антест.} {ПО для проведения автоматизированных измерений.}
\end{rSection} 

\end{document}
